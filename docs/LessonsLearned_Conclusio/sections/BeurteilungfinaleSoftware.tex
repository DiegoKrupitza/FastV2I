\section{Beurteilung finale Software}

\subsection{Gut geglückt}
Mithilfe von Swagger-Grouping war es möglich, alle Swagger Dokumentationen beim Gateway-Service zu aggregieren.
Da wir die asynchrone Kommunikation stark nutzen, haben wir kaum blockieren Anfragen was sich in der Performance des Projetkes widerspiegelt.
Durch die Microservicearchitektur waren wir in der Lage für simple Services (die nur CRUD Operation durchführen) leicht gewichtigte Frameworks und Technologien
einzusetzen. Für die komplexere Services konnten wir uns den Gebrauch von Technologien mit Threadsupport und OOP-Strategien zu nutzen machen.

\subsection{Verbesserungsmöglichkeiten}
Am Anfang des Projetkes wurde das Deployment nicht beachtet. Daher hatten wir für Service-Discovery ein Netflix-Eureka
Service genutzt. Beim Deployment ist uns dann aufgefallen, dass GKE dies für uns übernimmt. Dadurch mussten wir unsere
Service-Discovery-Logik und deren Integration im Gateway umschreiben. Aufgrund dieser gewonnenen Erfahrung ist es empfehlenswert,
bereits am Beginn der Entwicklung, das Deployment und Production-Environment zu beachten.

Die derzeitige Version von dem Simulator-Service simuliert die Erreignisse von den Autos und Ampeln. Es wäre möglich gewesen
den jetzigen Simulator-Service weiter in ein Auto-Simulator-Service, Ampel-Simulator-Service und Simulator-Manager aufzuspalten.


\section{Projektorganigramm}

\subsection{Rollen- und Aufgabenverteilung}

Unabhängig von den zugeteilten Rollen arbeiten alle Teammitglieder an allen Aspekten der Applikation.

\begin{itemize}
  \item Diego Krupitza ist \textit{Projektleiter} und \textit{Softwareentwickler}.
  \item Kian Pouresmaeil ist \textit{Softwarearchitekt} und \textit{-entwickler}.
  \item Jan Müller ist \textit{Frontend-Architekt} und \textit{Softwareentwickler}.
\end{itemize}

\subsection{Verantwortlichkeiten}

Alle Teammitglieder sind im Projekt gleichgestellt.
Die Rollenverteilung gilt als Grundlage für die Verteilung der Verantwortlichkeiten.

Beispielsweise ist der Frontend-Architekt Jan Müller verantwortlich für die vereinbarte \textit{Best-Practice-Implementierung} des Frontends.
Unabhängig von den Rollen ist jedes Teammitglied für die Implementierung aller Schichten verantwortlich.
Für die interne Koordination und Teamsteuerung ist Diego Krupitza verantwortlich.
Die Hauptaufgabe darin besteht, den Entwicklungsverlauf zu überwachen und die Arbeit der Teammitglieder zu koordinieren.

Aufgrund der flachen Struktur sind alle Teammitglieder gleichgestellt und es gibt keine Projektinterne Hierarchie.

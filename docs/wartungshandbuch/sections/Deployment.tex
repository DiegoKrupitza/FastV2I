\section{Deployment}

Für das Deployment wurden GKE sowie die dem Projekt beigelegten YAML-Dateien verwendet.
Die YAML-Dateien befinden sich im Verzeichnis \textit{kubernetes}.
Auf dem Kubernetes Cluster können diese Dateien mit dem Befehl \verb|kubectl apply -f .| angewendet werden. 
In den folgenden Unterpunkten finden sich detaillierte Beschreibungen der neun YAML-Konfigurations\-dateien.

\subsection{secrets-kubernetes.yaml}

Für das Deployment der Applikation auf Kubernetes müssen bestimmte sensible Information vorerst als \textit{Secrets} in das Cluster eingespielt werden.
Diese Secrets werden in einer Datei mit dem Namen \textit{secrets-kubernetes.yaml} abgelegt.
Da diese Secrets je nach Deployment unterscheiden können, gibt es edie Vorlage \textit{example-secrets-kubernetes.yaml}. 

Das Mapping der Secrets muss den Namen \textit{dse-secrets-v2i} haben.

Im Deployment befinden sich folgende Secrets.

\begin{table}[h]
	\begin{tabular}{|l|l|}
		\hline
		Name & Beschreibung \\ \hline
		\verb|MONGO_DB_HOST| &  Der Hostname der MongoDB Instanz  \\ \hline
		\verb|MONGO_DB_USER| &  Der Username der MongoDB Instanz  \\ \hline
		\verb|MONGO_DB_PWD|  &  Das Passwort der MongoDB Instanz  \\ \hline
	\end{tabular}
	\caption{Secrets des Deployments}
\end{table}

\subsection{ingress-kubernetes.yaml}

Für die Applikation wurde ein Ingress aufgesetzt, welches Anfragen basierend auf einem Pfadpräfix weiterleitet. 
Anfragen, deren Pfade mit \verb|/api/| beginnen, werden an den Gateway-Service weitergeleitet.
Jegliche andere Anfragen werden per Standardeinstellung an den Cockpit-Service weitergeleitet.

\subsection{gateway-kubernetes.yaml}

In dieser YAML-Datei liegt die Konfiguration des Gateways-Deployments vor.
Es wir ein Deployment mit dem Namen \textit{gateway} erstellt, welches das Docker Image \textit{deryeger/dse-gateway} verwendet.
Zusätzlich werden alle Konfigurationsvariablen, die in dem Kapitel \textit{Entwicklungsumgebung} beschrieben werden, gesetzt. 
Letztlich wird ein \textit{ClusterIP-Service} mit dem Namen \textit{gateway-service} erstellt, um den Port des Containers zugänglich zu machen.

\subsection{rabbitmq-kubernetes.yaml}

In dieser YAML-Datei liegt die Konfiguration einer RabbitMQ-Instanz vor.
Es wir ein Deployment mit dem Namen \textit{rabbitmq} erstellt, welches das Docker Image \textit{rabbitmq:management-alpine} verwendet. 
Letztlich wird ein ClusterIP-Service mit dem Namen \textit{rabbitmq-service} erstellt, um den Port des Containers zugänglich zu machen.

\subsection{cockpit-kubernetes.yaml}

In dieser YAML-Datei liegt die Konfiguration des Cockpit-Deployments vor.
Es wir ein deployment mit dem Namen \textit{cockpit} erstellt, welches das Docker Image \textit{deryeger/dse-cockpit} verwendet. 
Letztlich wird ein ClusterIP-Service mit dem Namen \textit{cockpit-service} erstellt, um den Port des Containers zugänglich zu machen.

\subsection{entity-kubernetes.yaml}

In dieser YAML-Datei liegt die Konfiguration des Entity-Service-Deployments vor.
Es wir ein Deployment mit dem Namen \textit{entity} erstellt, welches das Docker Image \textit{deryeger/dse-entity} verwendet.
Zusätzlich werden alle Konfigurationsvariablen, die in dem Kapitel \textit{Entwicklungsumgebung} beschrieben werden, gesetzt. 
Beim Setzen der Konfigurationsvariablen werden unter anderem Werte aus dem Secret \textit{dse-secrets-v2i} genutzt.
Letztlich wird ein ClusterIP-Service mit dem Namen \textit{entity-service} erstellt, um den Port des Containers zugänglich zu machen.

\subsection{simulator-kubernetes.yaml}

In dieser YAML-Datei liegt die Konfiguration des Simulator-Service-Deploy\-ments vor.
Es wir ein Deployment mit dem Namen \textit{simulator} erstellt, welches das Docker Image \textit{deryeger/dse-simulator} verwendet.
Zusätzlich werden alle Konfigurationsvariablen, die in dem Kapitel \textit{Entwicklungsumgebung} beschrieben werden, gesetzt. 
Letztlich wird ein ClusterIP-Service mit dem Namen \textit{simulator-service} erstellt, um den Port des Containers zugänglich zu machen.

\subsection{flowcontrol-kubernetes.yaml}

In dieser YAML-Datei liegt die Konfiguration des Flowcontrol-Service-Deploy\-ments vor.
Es wir ein Deployment mit dem Namen \textit{flowcontrol} erstellt, welches das Docker Image \textit{deryeger/dse-flowcontrol} verwendet.
Zusätzlich werden alle Konfigurationsvariablen, die in dem Kapitel \textit{Entwicklungsumgebung} beschrieben werden, gesetzt. 
Letztlich wird ein ClusterIP-Service mit dem Namen \textit{flowcontrol-service} erstellt welches den passenden Port des Containers zugänglich macht.

\subsection{tracking-kubernetes.yaml}

In dieser YAML-Datei liegt die Konfiguration des Tracking-Service-Deploy\-ments vor.
Es wir ein Deployment mit dem Namen \textit{tracking} erstellt, welches das Docker Image \textit{deryeger/dse-tracking} verwendet.
Zusätzlich werden alle Konfigurationsvariablen, die in dem Kapitel \textit{Entwicklungsumgebung} beschrieben werden, gesetzt. 
Beim Setzen der Konfigurationsvariablen werden unter anderem Werte aus dem Secret \textit{dse-secrets-v2i} genutzt.
Letztlich wird ein ClusterIP-Service mit dem Namen \textit{tracking-service} erstellt, um den Port des Containers zugänglich zu machen.

\section{Build Process}

Die Java-Dienste lassen über den Befehl \verb|mvn package -DskipTests| bauen.
Analog dazu baut der Befehl \verb|yarn build| die TypeScript-Dienste.

Letztendlich findet dies nur beim Bauen der Docker-Images statt, für welche pro Verzeichnis ein Dockerfile angelegt wurde.
Für eine kürzere Bauzeit wurden die Dockerfiles so programmiert, dass diese mehrere Ebenen verwenden und Dependencies zwischengespeichert.
Um das Bauen der Docker-Images weiter zu vereinfachen, können die \verb|dockerize|-Skripte der Datei \break \verb|package.json| verwendet werden.
Der lokale Docker-User muss dafür Zugriffsrechte auf die Docker Hub Repositories besitzen.

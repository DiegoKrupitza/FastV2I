\section{Frameworks und Bibliotheken}

In diesem Kapitel werden die verwendeten Frameworks und Bibliotheken kurz beschrieben.
Detaillierte Informationen können den Dokumentationen der jeweiligen Projekte entnommen werden.

\subsection{Entity-Service und Tracking-Service}

Sowohl der Entity-Service als auch der Tracking-Service wurden mithilfe des \href{https://www.fastify.io}{Fastify-Frameworks} entwickelt.
Dabei handelt es sich um ein Web-Framework für Node.js, welches sich durch seine hohe Performanz sowie Schema-Validierung auszeichnet.

Für die Kommunikation mit der MongoDB Datenbank wird die offizielle Bibliothek \href{https://github.com/mongodb/node-mongodb-native}{node-mongodb-native} verwendet.

Die Kommunikation mit der \enquote{MoM} basiert auf der Bibliothek \href{https://amqp-node.github.io/amqplib/}{amqplib}.

Darüber hinaus wird \href{https://github.com/motdotla/dotenv}{dotenv} zum Laden der Umgebungsvariablen verwendet.

\subsection{Cockpit}
JAN

\subsection{Flocontrol-Service}
DIEGO

\subsection{Simulator-Service}
DIEGO

\subsection{Gateway-Service}
DIEGO
